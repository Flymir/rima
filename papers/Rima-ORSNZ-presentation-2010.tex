%\documentclass[smaller,handout]{beamer}
\documentclass[smaller]{beamer}

\usepackage{listings}
\usepackage{textcomp}
\usepackage[normalem]{ulem}
\usepackage{verbatim}

\lstdefinelanguage{lua}
  {morekeywords={and,break,do,else,elseif,end,false,for,function,if,in,local,
     nil,not,or,repeat,return,then,true,until,while},
   morekeywords={[2]arg,assert,collectgarbage,dofile,error,_G,getfenv,
     getmetatable,ipairs,load,loadfile,loadstring,next,pairs,pcall,print,
     rawequal,rawget,rawset,select,setfenv,setmetatable,tonumber,tostring,
     type,unpack,_VERSION,xpcall},
   morekeywords={[2]coroutine.create,coroutine.resume,coroutine.running,
     coroutine.status,coroutine.wrap,coroutine.yield},
   morekeywords={[2]module,require,package.cpath,package.load,package.loaded,
     package.loaders,package.loadlib,package.path,package.preload,
     package.seeall},
   morekeywords={[2]string.byte,string.char,string.dump,string.find,
     string.format,string.gmatch,string.gsub,string.len,string.lower,
     string.match,string.rep,string.reverse,string.sub,string.upper},
   morekeywords={[2]table.concat,table.insert,table.maxn,table.remove,
   table.sort},
   morekeywords={[2]math.abs,math.acos,math.asin,math.atan,math.atan2,
     math.ceil,math.cos,math.cosh,math.deg,math.exp,math.floor,math.fmod,
     math.frexp,math.huge,math.ldexp,math.log,math.log10,math.max,math.min,
     math.modf,math.pi,math.pow,math.rad,math.random,math.randomseed,math.sin,
     math.sinh,math.sqrt,math.tan,math.tanh},
   morekeywords={[2]io.close,io.flush,io.input,io.lines,io.open,io.output,
     io.popen,io.read,io.tmpfile,io.type,io.write,file:close,file:flush,
     file:lines,file:read,file:seek,file:setvbuf,file:write},
   morekeywords={[2]os.clock,os.date,os.difftime,os.execute,os.exit,os.getenv,
     os.remove,os.rename,os.setlocale,os.time,os.tmpname},
   alsodigit = {.},
   sensitive=true,
   morecomment=[l]{--},
   morecomment=[s]{--[[}{]]--},
   morestring=[b]",
   morestring=[d]'
  }

\lstset{
  language=lua,
  basewidth=0.5em,
  basicstyle=\ttfamily\small,
  keywordstyle=\color{blue}\bfseries,
  keywordstyle=[2]\color{blue},
  keywordstyle=[3]\color{orange},
  commentstyle=\color[rgb]{0,.6,0},
%  stringstyle=\color{green},
  showstringspaces=false,
  morekeywords={[3]
    rima.R, rima.E, rima.repr, rima.sum, rima.sqrt, rima.nonnegative, rima.binary,
    rima.mp.new, rima.mp.C, rima.mp.solve},
  upquote=true,
  backgroundcolor=\color[rgb]{.9,.9,.9},
  escapeinside={(*}{*)},
  escapebegin=\bfseries\color{red},
}

\lstdefinelanguage{mosel}{
  morekeywords={[2] sum,in,forall,maximize,is_binary},
}

\lstdefinestyle{mosel}{
  language=mosel,
  basicstyle=\ttfamily\small,
}

\newcommand*\oldmacro{}%
\let\oldmacro\insertshorttitle%
\renewcommand*\insertshorttitle{%
  \oldmacro\hfill%
  \insertframenumber\,/\,\inserttotalframenumber}

\mode<presentation>
{
  \usetheme{Warsaw}
}

%%%%%%%%%%%%%%%%%%%%%%%%%%%%%%%%%%%%%%%%%%%%%%%%%%%%%%%%%%%%%%%%%%%%%%%%%%%%%%%%%%%%%%%%%%%%%%%%%%%%

\title[Rima]{Rima: Building Math Models for Reuse}
%\subtitle{Include Only If Paper Has a Subtitle}
\author[Geoff~Leyland]{Geoff~Leyland \\ \texttt{geoff.leyland@incremental.co.nz}}
\institute[Incremental Limited] {Incremental Limited}
\date[ORSNZ 2010]{ORSNZ 2010}


\begin{document}
%%%%%%%%%%%%%%%%%%%%%%%%%%%%%%%%%%%%%%%%%%%%%%%%%%%%%%%%%%%%%%%%%%%%%%%%%%%%%%%%%%%%%%%%%%%%%%%%%%%%

\begin{frame}
  \titlepage
\end{frame}


%%%%%%%%%%%%%%%%%%%%%%%%%%%%%%%%%%%%%%%%%%%%%%%%%%%%%%%%%%%%%%%%%%%%%%%%%%%%%%%%%%%%%%%%%%%%%%%%%%%%

\section{Model Reuse}

%%%%%%%%%%
\subsection{Model Reuse}
\begin{frame}{Composing Models from \emph{Reusable} Parts}


  {\bf Reuse:} Someone has written a model or model part,
  and someone would like to use the model or part in a different context
  \vspace{\stretch{1}}

  {\bf In this talk:}You have a model for a \emph{single} knapsack
  and you would like to extend it to a \emph{multiple} knapsack model (generalised assignment)
\end{frame}


%%%%%%%%%%
\subsection{Reusing a Knapsack}
\begin{frame}[fragile]{One to Many Knapsacks}
  We would like to fill a single sack with items of the highest value:
  \vspace{-2ex}
  \begin{overprint}
  \onslide<1| handout:0>
  \begin{lstlisting}[style=mosel]
(*maximize(sum(i in ITEMS) take(i) *\ value(i))*)
sum(i in ITEMS) take(i) * size(i) <= CAPACITY
forall(i in ITEMS) take(i) is_binary
  \end{lstlisting}

  \onslide<2| handout:0>
  \begin{lstlisting}[style=mosel]
maximize(sum(i in ITEMS) take(i) * value(i))
(*sum(i in ITEMS) take(i) *\ size(i) <= CAPACITY*)
forall(i in ITEMS) take(i) is_binary
  \end{lstlisting}

  \onslide<3| handout:0>
  \begin{lstlisting}[style=mosel]
maximize(sum(i in ITEMS) take(i) * value(i))
sum(i in ITEMS) take(i) * size(i) <= CAPACITY
(*forall(i in ITEMS) take(i) is\_binary*)
\end{lstlisting}

  \onslide<4-| handout:1>
  \begin{lstlisting}[style=mosel]
maximize(sum(i in ITEMS) take(i) * value(i))
sum(i in ITEMS) take(i) * size(i) <= CAPACITY
forall(i in ITEMS) take(i) is_binary
  \end{lstlisting}
  \vspace{2ex}

  Now we'd like to extend the model to cover several sacks:
  \begin{lstlisting}[style=mosel]
maximize(
  sum((*s in SACKS,*) i in ITEMS) take((*s,*) i) * value(i))
(*forall(s in SACKS)*)
  sum(i in ITEMS) take((*s,*) i) * size(i) <= CAPACITY(*(s)*)
forall((*s in SACKS,*) i in ITEMS) take((*s,*) i) is_binary
(*\color{orange}forall(i in ITEMS) sum(s in SACKS) take(s, i) <= 1*)
\end{lstlisting}
  \end{overprint}
\end{frame}


%%%%%%%%%%
\subsection{Why do we care?}
\begin{frame}{Why do we care?}
  It's only a few extra characters, right?
  \vspace{\stretch{1}}\pause
  \begin{itemize}
    \item We might have a more complex model than a knapsack
    \vspace{1ex}
    \item Altering the text of the model makes it hard to re-use
      \begin{itemize}
        \item to re-use it, we have to understand it enough to modify it
        \item it is hard to share improvements between the two models
      \end{itemize}
  \end{itemize}
\end{frame}


%%%%%%%%%%
\begin{frame}
  Why program by hand in five days what you can spend five years of your life automating?\\
  ~\\
  - Terence Parr
\end{frame}


%%%%%%%%%%
\subsection{What is Rima?}
\begin{frame}{What is Rima?}
  Rima:
  \begin{itemize}
    \item is \emph{Yet-Another} Math Programming Modelling Language
    \item focuses on making it easy to construct and re-use models
  \end{itemize}
  \vspace{\stretch{1}}
  \begin{itemize}
    \item is MIT licensed and available at {\tt http://rima.googlecode.com/}
    \item is implemented in Lua: {\tt http://www.lua.org/}
    \begin{itemize}\item a small, fast ``scripting'' language\end{itemize}
    \item currently binds to CLP, CBC and lpsolve
    \item has been submitted to COIN for review
  \end{itemize}
\end{frame}


%%%%%%%%%%%
\begin{frame}{Contents}
  \textbf{Algorithms + Data Structures = Programs}\\
  ~\\
  - Niklaus Wirth\\
  \vspace{\stretch{1}}
  \small\textbf{Symbolic Expressions + Structured Data = Reusable Model Components}\\ 
\end{frame}
  


%%%%%%%%%%%%%%%%%%%%%%%%%%%%%%%%%%%%%%%%%%%%%%%%%%%%%%%%%%%%%%%%%%%%%%%%%%%%%%%%%%%%%%%%%%%%%%%%%%%%

%\begin{frame}
%  \frametitle{Outline}
%  \tableofcontents
%\end{frame}


%%%%%%%%%%%%%%%%%%%%%%%%%%%%%%%%%%%%%%%%%%%%%%%%%%%%%%%%%%%%%%%%%%%%%%%%%%%%%%%%%%%%%%%%%%%%%%%%%%%%

\section{Symbolic Expressions}
\begin{comment}
%%%%%%%%%%
\subsection{Expressions}
\begin{frame}{Symbolic Expressions}
  In Rima, the objective and constraints are stored as symbolic expressions\\
  ~\\
  Expressions and data are combined and into matrix rows at solve time
  \\
  ~\\
  \begin{itemize}
    \item dedicated modelling languages tend to do the same
    \item some ``mainstream language'' bindings directly construct matrix rows
 \end{itemize}
\end{frame}
\end{comment}

%%%%%%%%%%
\begin{frame}[fragile]{Constructing Expressions}

  Rima expressions involve \emph{references} constructed with \lstinline!rima.R!:
  \begin{lstlisting}
e = rima.R("a") * rima.R("x") + rima.R("b") * rima.R("y")
  \end{lstlisting}
  \vspace{\stretch{1}}\pause

  Expressions are stored symbolically and print nicely:
  \begin{lstlisting}
print(e)                           --> a*x + b*y
  \end{lstlisting}
%  {``\footnotesize\lstinline!--!'' introduces a comment in Lua.
%   I will use ``\lstinline!-->!'' to show output}
  \vspace{\stretch{1}}\pause
  

  All the \lstinline!rima.R!'s are cumbersome, so there is a shortcut:
  \begin{lstlisting}
(*rima.define("a, x, b, y")*)
e = a * x + b * y
print(e)                           --> a*x + b*y
  \end{lstlisting}
  \vspace{\stretch{1}}\pause

  You can manipulate expressions like references:
  \begin{lstlisting}
print(3 * e)                       --> 3*(a*x + b*y)
print(e^2)                         --> (a*x + b*y)^2
  \end{lstlisting}
\end{frame}


%%%%%%%%%%
\begin{frame}[fragile]{Evaluating Expressions}

  \lstinline!rima.E! evaluates expressions by matching references to a \emph{table} of values:
  \begin{lstlisting}
rima.define("a, x, b, y")
e = a * x + b * y
print((*rima.E(e, \verb|{|a=2,x=3,b=4,y=5\verb|}|)*))--> 26
  \end{lstlisting}
  \vspace{\stretch{1}}\pause

  If some references are undefined, \lstinline!rima.E! returns a new expression:
  \begin{lstlisting}
print(rima.E(e, {a=2,b=4}))        --> 2*x + 4*y
  \end{lstlisting}
  \vspace{\stretch{1}}\pause

 The values you provide as data to \lstinline!rima.E! can be other expressions:
  \begin{lstlisting}
rima.define("xpos, xneg")
print(rima.E(e, {(*\verb!x=xpos - xneg!*)})   --> a*(xpos - xneg) + b*y
  \end{lstlisting}
\end{frame}


%%%%%%%%%%
\subsection{A Simple LP}
\begin{frame}[fragile]{A Simple LP (1)}

  \lstinline!rima.mp.new! creates a model, and \lstinline!rima.mp.C! builds a constraint:
  \vspace{-2ex}
  \begin{overprint}
  \onslide<1| handout:0>
  \begin{lstlisting}
rima.define("a, b, x, y")

M = rima.mp.new({
  sense = "maximise",
  objective = a*x + b*y,
  \end{lstlisting}

  \onslide<2| handout:0>
  \begin{lstlisting}
rima.define("a, b, x, y")

M = rima.mp.new({
  sense = "maximise",
  objective = a*x + b*y,

  C1 = rima.mp.C(x + 2*y, "<=", 3),
  C2 = rima.mp.C(2*x + y, "<=", 3),
  \end{lstlisting}

  \onslide<3| handout:1>
  \begin{lstlisting}
rima.define("a, b, x, y")

M = rima.mp.new({
  sense = "maximise",
  objective = a*x + b*y,

  C1 = rima.mp.C(x + 2*y, "<=", 3),
  C2 = rima.mp.C(2*x + y, "<=", 3),

  x = rima.positive(),
  y = rima.positive()
})
  \end{lstlisting}

  \end{overprint}
\end{frame}


%%%%%%%%%%
\begin{frame}[fragile]{A Simple LP (2)}
  As with expressions, you can print the model:
  \begin{lstlisting}
print(M)
--> Maximise:
-->   a*x + b*y
--> Subject to:
-->   C1: x + 2*y <= 3
-->   C2: 2*x + y <= 3
-->   0 <= x <= inf, x real
-->   0 <= y <= inf, x real
  \end{lstlisting}
\end{frame}


%%%%%%%%%%
\begin{frame}[fragile]{A Simple LP (3)}

  \lstinline{rima.mp.solve} takes the model and a table of data and solves:
  \begin{lstlisting}
primal, dual = rima.mp.solve("clp", M, {a=2, b=2})
print(primal.objective)             --> 4
print(primal.x)                     --> 1
print(primal.y)                     --> 1
print(primal.C1)                    --> 3

print(dual.x)                       --> 0
print(dual.C1)                      --> 0.333
  \end{lstlisting}
  \vspace{\stretch{1.0}}
  \bf{\lstinline{M} encapsulates a complete, symbolic representation of the model}
\end{frame}


%%%%%%%%%%%%%%%%%%%%%%%%%%%%%%%%%%%%%%%%%%%%%%%%%%%%%%%%%%%%%%%%%%%%%%%%%%%%%%%%%%%%%%%%%%%%%%%%%%%%

\section{Structured Data}

\begin{comment}
%%%%%%%%%%
\begin{frame}{Structured Data}
  Rima data can be richly structured
  \begin{itemize}
    \item like all other languages, Rima supports arrays
    \item Rima also supports structures
    \begin{itemize}
      \item most general-purpose programming languages have structures
      \item some dedicated modelling languages \emph{do not} have structures
      \item most language bindings support structures in their host language, but the modelling systems themselves have weaker support
    \end{itemize}
 \end{itemize}
\end{frame}
\end{comment}

%%%%%%%%%%
\subsection{Arrays, Sums and Array Assignment}
\begin{frame}[fragile]{Arrays, Sums and Array Assignment}

  You can index references as if they were arrays:%
  \vspace{-2ex}
  \begin{overprint}%
%
  \onslide<1| handout:0>%
  \begin{lstlisting}
rima.define("X")
e = X[1] + X[2] + X[3]
print(e)                           --> X[1] + X[2] + X[3]
  \end{lstlisting}%
%
  \onslide<2-| handout:1>%
  \begin{lstlisting}
rima.define("X")
e = X[1] + X[2] + X[3]
print(e)                           --> X[1] + X[2] + X[3]
print(rima.E(e, {(*\verb!X={1,2,3}!*)}))      --> 6
  \end{lstlisting}
%  
  \end{overprint} 
  \vspace{\stretch{1}}\pause\pause
  \vspace{-1ex}

  \lstinline!rima.sum! sums an expression over a set:
  \begin{lstlisting}
rima.define("x, X")
e = (*\verb!rima.sum{x=X}(x^2)!*)
print(rima.E(e, {X={1,2,3}}))      --> 14
  \end{lstlisting}
  \vspace{\stretch{1}}\pause

  You can assign to a whole array at once:
  \begin{lstlisting}
rima.define("i, X")
(*\verb!t = { [X[i]] = 2^i }!*)
print(rima.E(X[5], t))             --> 32
  \end{lstlisting}

\end{frame}


%%%%%%%%%%
\subsection{Structures}
\begin{frame}[fragile]{Structures}

  As well as using arrays, you can also index references as if they were structures:
  \begin{lstlisting}
rima.define("item")
mass = item.(*volume*) * item.(*density*)
print(mass)
  --> item.volume * item.density
print(rima.E(mass, {(*\verb!item={volume=10, density=1.032}!*)}))
  --> 10.32
  \end{lstlisting}
  \vspace{\stretch{1}}
\end{frame}


%%%%%%%%%%
\subsection{A Structured Knapsack}
\begin{frame}[fragile]{A Structured Knapsack (1)}
  \vspace{-2ex}
  \begin{overprint}%
%
  \onslide<1| handout:0>%
  \begin{lstlisting}
rima.define("i, items")      -- items in knapsack
rima.define("capacity")

knapsack = rima.mp.new()
 
(*\verb!knapsack.sense = "maximise"!*)
(*\verb!knapsack.objective = rima.sum{i=items}(i.take * i.value)!*)

knapsack.capacity_limit = rima.mp.C(
  rima.sum{i=items}(i.take * i.size), "<=", capacity)

knapsack.items[{i=items}].take = rima.binary()
  \end{lstlisting}

  \bf{Remember this model, because we won't change it from here on}

  \onslide<2| handout:0>%
  \begin{lstlisting}
rima.define("i, items")      -- items in knapsack
rima.define("capacity")

knapsack = rima.mp.new()
 
knapsack.sense = "maximise"
knapsack.objective = rima.sum{i=items}(i.take * i.value)!

(*\verb!knapsack.capacity_limit = rima.mp.C(!*)
(*\verb!  rima.sum{i=items}(i.take * i.size), "<=", capacity)!*)

knapsack.items[{i=items}].take = rima.binary()
  \end{lstlisting}

  \bf{Remember this model, because we won't change it from here on}

  \onslide<3| handout:0>%
  \begin{lstlisting}
rima.define("i, items")      -- items in knapsack
rima.define("capacity")

knapsack = rima.mp.new()
 
knapsack.sense = "maximise"
knapsack.objective = rima.sum{i=items}(i.take * i.value)

knapsack.capacity_limit = rima.mp.C(
  rima.sum{i=items}(i.take * i.size), "<=", capacity)

(*\verb!knapsack.items[{i=items}].take = rima.binary()!*)
  \end{lstlisting}

  \bf{Remember this model, because we won't change it from here on}

  \onslide<4| handout:1>%
  \begin{lstlisting}
rima.define("i, items")      -- items in knapsack
rima.define("capacity")

knapsack = rima.mp.new()
 
knapsack.sense = "maximise"
knapsack.objective = rima.sum{i=items}(i.take * i.value)

knapsack.capacity_limit = rima.mp.C(
  rima.sum{i=items}(i.take * i.size), "<=", capacity)

knapsack.items[{i=items}].take = rima.binary()
  \end{lstlisting}

  \bf{Remember this model, because we won't change it from here on}
  \end{overprint}
\end{frame}


%%%%%%%%%%
\begin{frame}[fragile]{A Structured Knapsack (2)}
  As usual, rima can write the model out for us:
  \begin{lstlisting}
print(rima.repr(knapsack, {format="latex"}))
  \end{lstlisting}
  \vspace{\stretch{1}}

  In \LaTeX:
\begin{align*}
\text{\bf maximise} & \sum_{i \in \text{items}} i_{\text{take}} i_{\text{value}} \\
\text{\bf subject to} \\
\text{capacity\_limit}: & \sum_{i \in \text{items}} i_{\text{size}} i_{\text{take}} \leq \text{capacity} \\
& \text{items}_{i,\text{take}} \in \{0, 1\} \forall i \in \text{items}\\
\end{align*}

\end{frame}



%%%%%%%%%%
\begin{frame}[fragile]{A Structured Knapsack (3)}
  \begin{lstlisting}
ITEMS = {
  camera   = { value =  15, size =  2 },
  necklace = { value = 100, size = 20 },
  vase     = { value =  15, size = 20 },
  picture  = { value =  15, size = 30 },
  tv       = { value =  15, size = 40 },
  video    = { value =  15, size = 30 },
  chest    = { value =  15, size = 60 },
  brick    = { value =   1, size = 10 }}

primal = rima.mp.solve("cbc", knapsack,
  {items=ITEMS, capacity=102})

print(primal.objective)             --> 160
print(primal.items.camera.take)     --> 1
print(primal.items.vase.take)       --> 1
print(primal.items.brick.take)      --> 0
  \end{lstlisting}
\end{frame}


%%%%%%%%%%%%%%%%%%%%%%%%%%%%%%%%%%%%%%%%%%%%%%%%%%%%%%%%%%%%%%%%%%%%%%%%%%%%%%%%%%%%%%%%%%%%%%%%%%%%

\section{Reusable Model Components}
\begin{comment}
%%%%%%%%%%
\begin{frame}{Reusable Model Components}
  With Rima we can build a model from parts by:
  \begin{itemize}
    \item Configuring an existing model
    \item Extending an existing model
    \item Including an existing model, perhaps more than once
 \end{itemize}
\end{frame}
\end{comment}

%%%%%%%%%%
\subsection{Extensible Models}
\begin{frame}[fragile]{Constraints are Data Too}
  Suppose, for example, you can only pick one of the camera or the vase.\\
  \vspace{\stretch{1}}

  Constraints, like expressions, are just data, so modelling this is easy:
  \begin{lstlisting}
primal = rima.mp.solve("cbc", knapsack,
  {items=ITEMS, capacity=102,
  (*\verb!camera_xor_vase =!*)
    (*rima.mp.C(items.camera.take + items.vase.take, "<=", 1)*)})
  
print(primal.objective)             --> 146
print(primal.items.camera.take)     --> 1
print(primal.items.vase.take)       --> 0
  \end{lstlisting}
  \vspace{\stretch{1}}
  \bf{What if we want to reuse this constrained model?}
\end{frame}


%%%%%%%%%%
\begin{frame}[fragile]{Extensible Models}
  \lstinline!rima.mp.new! can take two arguments,
  the model you want to extend
  and any extensions to the model:
  \begin{lstlisting}
side_constrained_knapsack = rima.mp.new(knapsack, {
  (*\verb!camera_xor_vase =!*)
    (*\verb!rima.mp.C(items.camera.take + items.vase.take, "<=", 1)!*)})

primal = rima.mp.solve("cbc", side_constrained_knapsack,
  {items=ITEMS, capacity=102})

print(primal.objective)             --> 146
  \end{lstlisting}
  \vspace{\stretch{1}}
\end{frame}


%%%%%%%%%%
\subsection{Multiple Sacks}
\begin{frame}[fragile]{Multiple Sacks}
  Now we are ready to try a multiple sack model:
  \vspace{-2ex}
  \begin{overprint}%
%
  \onslide<1| handout:0>%
  \begin{lstlisting}
rima.define("s, sacks")

multiple_sack = rima.mp.new({
(*\verb!  sense = "maximise",!*)
(*\verb!  objective = rima.sum{s=sacks}(s.objective),!*)
  only_take_once[{i=items}] =
    rima.mp.C(rima.sum{s=sacks}(s.items[i].take), "<=", 1)
})
  \end{lstlisting}
  \vspace{\stretch{1}}
  Note that:
  \begin{itemize}
    \item we are treating \lstinline!sacks! like a ``substructure''
    \item we have not said anything about what \lstinline!sacks! is
   \end{itemize}

  \onslide<2| handout:0>%
  \begin{lstlisting}
rima.define("s, sacks")

multiple_sack = rima.mp.new({
  sense = "maximise",
  objective = rima.sum{s=sacks}(s.objective),
(*\verb!  only_take_once[{i=items}] =!*)
(*\verb!    rima.mp.C(rima.sum{s=sacks}(s.items[i].take), "<=", 1)!*)
})
  \end{lstlisting}
  \vspace{\stretch{1}}
  Note that:
  \begin{itemize}
    \item we are treating \lstinline!sacks! like a ``substructure''
    \item we have not said anything about what \lstinline!sacks! is
   \end{itemize}

  \onslide<3| handout:1>%
  \begin{lstlisting}
rima.define("s, sacks")

multiple_sack = rima.mp.new({
  sense = "maximise",
  objective = rima.sum{s=sacks}(s.objective),
  only_take_once[{i=items}] =
    rima.mp.C(rima.sum{s=sacks}(s.items[i].take), "<=", 1)
})
  \end{lstlisting}
  \vspace{\stretch{1}}
  Note that:
  \begin{itemize}
    \item we are treating \lstinline!sacks! like a ``substructure''
    \item we have not said anything about what \lstinline!sacks! is
   \end{itemize}

  \end{overprint}
\end{frame}


%%%%%%%%%%
\begin{frame}[fragile]{Multiple Knapsacks}
  We can specify what the knapsack submodel is when we solve:

  \begin{lstlisting}
primal = rima.mp.solve("cbc", multiple_sack, {
  items = ITEMS,
  [sacks[s].items] = items,
  sacks = {{capacity=51}, {capacity=51}},
  (*[sacks[s]] = knapsack*)})

print(primal.objective)             --> 146
  \end{lstlisting}
  Sack 1: \lstinline!camera, vase, brick!\\
  Sack 2: \lstinline!necklace, video!
\end{frame}


%%%%%%%%%%
\begin{frame}[fragile]{Multiple \emph{Constrained} Knapsacks}
  What if we can't carry the camera and vase in the same sack?

  \begin{lstlisting}
primal = rima.mp.solve("cbc", multiple_sack, {
  items = ITEMS,
  [sacks[s].items] = items,
  sacks = {{capacity=51}, {capacity=51}},
  [sacks[s]] = (*\verb!side_constrained_knapsack!*)})

print(primal.objective)             --> 146
  \end{lstlisting}
  Sack 1: \lstinline!camera, picture, brick!\\
  Sack 2: \lstinline!vase, necklace!
\end{frame}


%%%%%%%%%%%%%%%%%%%%%%%%%%%%%%%%%%%%%%%%%%%%%%%%%%%%%%%%%%%%%%%%%%%%%%%%%%%%%%%%%%%%%%%%%%%%%%%%%%%%
\section{Conclusion}

%%%%%%%%%%
\begin{frame}{Conclusion}
  We wrote a knapsack model and reused it without any modification:
  \begin{itemize}
    \item in a side-constrained knapsack
    \item as a part of a multiple knapsack problem
    \item as a part of a multiple side-constrained knapsack problem
  \end{itemize}
  \vspace{\stretch{1}}

  Structured symbolic models enable reuse \emph{without} alteration:
  \begin{itemize}
    \item we only need to understand the model interface to reuse it
    \item it is easy to share improvements
  \end{itemize}

  \vspace{\stretch{1}}
  {\bf Can we ``compose'' complex models, and is it worthwhile?}
\end{frame}


%%%%%%%%%%%%%%%%%%%%%%%%%%%%%%%%%%%%%%%%%%%%%%%%%%%%%%%%%%%%%%%%%%%%%%%%%%%%%%%%%%%%%%%%%%%%%%%%%%%%
\end{document}
